%\clearpage
\subsection{Requisiti Funzionali} \label{subsec: requisitiF}
\normalsize
\begin{longtable}{|l|m{11cm}|}
\hline
\textbf{Id Requisito} & \textbf{Descrizione}\\
\hline
\endhead
RFO1 & L'utente deve poter visualizzare la lista dei template offerti dall'applicazione \\ \hline
RFO1.1 & L'utente deve poter visualizzare le miniature dei template \\ \hline
RFF1.1.1 & La comparsa delle miniature all'interno della lista deve avvenire in modo animato \\ \hline
RFO1.2 & L'utente deve poter scorrere la lista dei template \\ \hline
RFO1.3 & L'utente deve poter visualizzare la categoria dei template \\ \hline
RFO1.4 & L'utente deve poter selezionare i template \\ \hline
RFD1.4.1 & La selezione del template deve avvenire in modo animato \\ \hline
RFO2 & L'utente deve visualizzare il template selezionato in un apposito view-box \\ \hline
RFF2.1 & La comparsa del template nel view-box deve avvenire in modo animato \\ \hline
RFD2.2 & Se il template contiene plug-in JQuery l'utente deve poter visualizzare l'esecuzione del plug-in \\ \hline
RFO2.3 & L'utente deve visualizzare nel view-box l'effetto delle modifiche appotrate al template \\ \hline
RFO2.3.1 & L'utente deve poter visualizzare l'effetto sul template delle modifiche ai dati di tipo colore \\ \hline
RFO2.3.2 & L'utente deve poter visualizzare l'effetto sul template delle modifiche ai dati di tipo numero \\ \hline
RFO2.3.3 & L'utente deve poter visualizzare l'effetto sul template delle modifiche ai dati di tipo booleano \\ \hline
RFO2.3.4 & L'utente deve poter visualizzare l'effetto sul template delle modifiche ai dati di tipo immagine \\ \hline
RFO2.3.5 & L'utente deve poter visualizzare l'effetto sul template delle modifiche ai dati di tipo url \\ \hline
RFO2.3.6 & L'utente deve poter visualizzare l'effetto sul template delle modifiche ai dati di tipo mail\\ \hline
RFO2.3.7 & L'utente deve poter visualizzare l'effetto sul template delle modifiche ai dati di tipo stringa breve\\ \hline
RFO2.3.8 & L'utente deve poter visualizzare l'effetto sul template delle modifiche ai dati di tipo testo \\ \hline
RFO2.3.9 & L'utente deve poter visualizzare l'effetto sul template delle modifiche ai dati di tipo JSON \\ \hline
RFO3 & L'utente deve poter visualizzare i dati di default forniti dal template selezionato \\ \hline
RFO3.1 & L'utente deve poter visualizzare i dati di tipo colore \\ \hline
RFO3.2 & L'utente deve poter visualizzare i dati di tipo numero \\ \hline
RFO3.3 & L'utente deve poter visualizzare i dati di tipo booleano \\ \hline
RFO3.4 & L'utente deve poter visualizzare i dati di tipo immagine \\ \hline
RFO3.5 & L'utente deve poter visualizzare i dati di tipo url \\ \hline
RFO3.6 & L'utente deve poter visualizzare i dati di tipo mail \\ \hline
RFO3.7 & L'utente deve poter visualizzare i dati di tipo stringa breve \\ \hline
RFO3.8 & L'utente deve poter visualizzare i dati di tipo testo \\ \hline
RFO3.9 & L'utente deve poter visualizzare i dati in formato JSON \\ \hline
RFO4 & L'utente deve poter modificare i dati di default forniti dal template selezionato \\ \hline
RFO4.1 & L'utente deve poter modificare i dati di tipo colore \\ \hline
RFD4.1.1 & L'utente deve poter visualizzare il color-picker \\ \hline
RFF4.1.1.1 & La comparsa del color-picker deve avvenire in modo animato \\ \hline
RFF4.1.1.2 & La scomparsa del color-picker deve avvenire in modo animato \\ \hline
RFD4.1.2 & L'utente deve poter selezionare il colore nel color-picker \\ \hline
RFO4.2 & L'utente deve poter modificare i dati di tipo numero \\ \hline
RFO4.3 & L'utente deve poter modificare i dati di tipo booleano \\ \hline
RFO4.4 & L'utente deve poter modificare i dati di tipo immagine \\ \hline
RFD4.4.1 & L'utente deve poter caricare un'immagine da filesystem \\ \hline
RFO4.5 & L'utente deve poter modificare i dati di tipo url \\ \hline
RFO4.6 & L'utente deve poter modificare i dati di tipo mail \\ \hline
RFO4.7 & L'utente deve poter modificare i dati di tipo stringa breve \\ \hline
RFO4.8 & L'utente deve poter modificare i dati di tipo testo \\ \hline
RFO4.9 & L'utente deve poter modificare i dati dell'oggetto in formato JSON \\ \hline
RFD5 & L'utente deve poter visualizzare un messaggio di errore nel caso in cui l'inserimento dei dati avvenga in maniera non corretta \\ \hline

\caption[Requisiti Funzionali]{Requisiti Funzionali}
\label{tabella:req0}
\end{longtable}
\clearpage
\subsection{Requisiti di Vincolo}
\normalsize
\begin{longtable}{|l|m{11cm}|}
\hline
\textbf{Id Requisito} & \textbf{Descrizione} \\
\hline
\endhead
RVO1 & L'applicazione deve utilizzare HTML5 \\ \hline
RVO2 & L'applicazione deve utilizzare CSS3 \\ \hline
RVO3 & L'applicazione deve utilizzare JavaScript \\ \hline
RVO4 & L'applicazione deve utilizzare Ractive.js \\ \hline
RVO5 & L'applicazione deve funzionare su \textit{Google Chrome} versione 52.0 o superiore \\ \hline
RVO6 & L'applicazione deve funzionare su \textit{Firefox} versione 46.0 o superiore \\ \hline
RVD7 & L'applicazione deve funzionare su \textit{Safari} versione 9.0 o superiore \\ \hline
RVD8 & L'applicazione deve funzionare su \textit{Edge} versione 37.0 o superiore \\ \hline
RVD9 & L'applicazione deve funzionare su \textit{Opera} versione 37.0 o superiore \\ \hline
RVF10 & L'applicazione deve funzionare su \textit{Internet Explorer} versione 11.0 o superiore \\ \hline
\caption[Requisiti di Vincolo]{Requisiti di Vincolo}
\label{tabella:req1}
\end{longtable}
