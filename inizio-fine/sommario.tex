% !TEX encoding = UTF-8
% !TEX TS-program = pdflatex
% !TEX root = ../tesi.tex
% !TEX spellcheck = it-IT

%**************************************************************
% Sommario
%**************************************************************
\cleardoublepage
\phantomsection
\pdfbookmark{Sommario}{Sommario}
\begingroup
\let\clearpage\relax
\let\cleardoublepage\relax
\let\cleardoublepage\relax

\chapter*{Sommario}

Il presente documento descrive il lavoro svolto durante il periodo di stage, della durata di trecentoventi ore, dal laureando \myName presso l'azienda \myCompany.
L'obiettivo di tale attività di stage è l'analisi di varie librerie Javascript per la realizzazione di template HTML, al fine di poter realizzare un editor grafico che permetta la selezione e la modifica dei template per un loro sucessivo inserimento all'interno di pagine HTML.
Inoltre è stato effettuato uno studio sul comportamento dei template in ambito responsive, sulla possibilità di inserire plug-in jQuery all'interno dei template e su di un metodo di caricamento delle librerie controllato in modo di non avere più istanze della stessa libreria se utilizzata da diversi template.
%\vfill
%
%\selectlanguage{english}
%\pdfbookmark{Abstract}{Abstract}
%\chapter*{Abstract}
%
%\selectlanguage{italian}

\endgroup			

\vfill

