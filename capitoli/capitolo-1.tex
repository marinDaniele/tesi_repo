% !TEX encoding = UTF-8
% !TEX TS-program = pdflatex
% !TEX root = ../tesi.tex
% !TEX spellcheck = it-IT

%*************************************************************
\chapter{Introduzione}
\label{cap:introduzione}
%*************************************************************

\section{L'azienda}

\begin{figure}[htp]
\centering
\includegraphics[width=\textwidth/2]{../immagini/logo_zucchetti}
\caption{Logo di Zucchetti S.p.a.}
\end{figure}

La Zucchetti s.p.a. è una software house con sede a Lodi, che si occupa di soluzioni complete per le aziende, professionisti(commercialisti, consulenti del lavoro, avvocati, curatori fallimentari, notai ecc.) e pubbliche amministrazioni(Comuni, Province, Regioni, Ministeri, società pubbliche ecc.).\\
Il gruppo Zucchetti è la prima azzienda italiana in Europa con oltre 3300 addetti, 1100 partner e oltre 105000 clienti.\\
Le soluzioni principali proposte dall'azienda sono :\\
\begin{itemize}
\item Softwere: gestionali, per la sicurezza sul lavoro, analisi business ecc.
\item Hardware: per la rilevazione presenze, controllo accessi e controllo produzione.
\item Servizi: di outsourcing, cloud computing e data center.
\end{itemize}

\subsection{Portal Studio}
Lo stage si è svolto nella sede distaccata di Padova che si occupa di ricerca e sviluppo.
Tra i software che vengono sviluppati in questa sede è presente Portal Studio che consiste in una WEB application per la creazione di siti web.\\
L'applicazione offre all'utente un set completo di strumenti per la creazione di pagine web, permette la creazione e modifica in modo grafico della struttura HTML, la gestione degli stili tramite editor grafico per il CSS ed inoltre permette di gestire i dati provenienti da diversi tipi di database, il loro filtraggio e il binding con varie strutture HTML come liste e tabelle.\\
Il software risulta essere molto maturo e oltre alle funzionalità sopracitate permette anche la creazione di portlet e pagelet e altri elementi riutilizzabili e la gestione di risorse come dati in formato JSON.


\section{Il progetto}
