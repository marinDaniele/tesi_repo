% !TEX encoding = UTF-8
% !TEX TS-program = pdflatex
% !TEX root = ../tesi.tex
% !TEX spellcheck = it-IT

%*************************************************************
\chapter{Introduzione}
\label{cap:introduzione}
%*************************************************************

\section{L'azienda}

\begin{figure}[htp]
\centering
\includegraphics[width=\textwidth/2]{../immagini/logo_zucchetti}
\caption{Logo di Zucchetti S.p.a.}
\end{figure}

La \myCompany è una software house con sede a Lodi, che si occupa di soluzioni complete per le aziende, professionisti(commercialisti, consulenti del lavoro, avvocati, curatori fallimentari, notai ecc.) e pubbliche amministrazioni(Comuni, Province, Regioni, Ministeri, società pubbliche ecc.).\\
Il gruppo Zucchetti è la prima azzienda italiana in Europa con oltre 3300 addetti, 1100 partner e oltre 105000 clienti.\\
Le soluzioni principali proposte dall'azienda sono :\\
\begin{itemize}
\item Softwere: gestionali, per la sicurezza sul lavoro, analisi business ecc.
\item Hardware: per la rilevazione presenze, controllo accessi e controllo produzione.
\item Servizi: di outsourcing, cloud computing e data center.
\end{itemize}

\subsection{Portal Studio}
Lo stage si è svolto nella sede distaccata di Padova che si occupa di ricerca e sviluppo.
Tra i software che vengono sviluppati in questa sede è presente Portal Studio che consiste in una WEB application per la creazione di siti web.\\
L'applicazione offre all'utente un set completo di strumenti per la creazione di pagine web, permette la creazione e modifica in modo grafico della struttura HTML, la gestione degli stili tramite editor grafico per il CSS ed inoltre permette di gestire i dati provenienti da diversi tipi di database, il loro filtraggio e il binding con varie strutture HTML come liste e tabelle.\\
Il software risulta essere molto maturo e oltre alle funzionalità sopracitate permette anche la creazione di portlet e pagelet e altri elementi riutilizzabili e la gestione di risorse come dati in formato JSON.


\section{Il progetto}
Il progetto proposto dall'azienda per lo stage, nasce dal desiderio di aggiungere all'applicazione Portal Studio una nuova funzionalità che consiste nell'offrire all'utente la possibilità di inserire nelle proprie pagine HTML dei template già pronti e selezionabili da un insieme prestabilito.\\
Questo desiderio ha portato l'azienda ad interessarsi ai template engine come Mustache.js, HandlebarJS ecc.\\
Lo stage si divide in due parti.\\
La prima consisteva nello studio dei template e degli aspetti ad essi correlati, la seconda nella realizzazione di un editor che ne permettesse la visualizzazione e la modifica.
\subsection{Prima parte}
La prima parte del progetto inizia con la realizzazione di qualche template prototipo, utile sia per studiare le possibilità della libreria scelta sia per avere un insieme di template da inserire nell'editor che è stato realizzato in seguito.\\
Durante questa parte del progetto l'attenzione è stata rivolta alla possibilità di realizzare template statici, dinamici, template come composizione di altri template (es. lista di contatti) e template che utilizzano SVG.\\
In seguito alla realizzazione dei template prototipo è stato effettuato uno studio sulla possibilità di rendere i template responsive cioè permetterne la visualizzazione sia su dispositivi desktop che mobile.\\
La prima parte si è conclusa con uno studio sulla possibilità di realizzare template che contenessero al loro interno plug-in JQuery e sulla gestione del caricamento delle librerie per il loro funzionamento all'interno della pagina HTML in cui vengono inseriti.
\subsection{Seconda parte}
La seconda parte del progetto consisteva nella realizzazione di un editor grafico che permettesse all'utente la selezione di un template da una lista fornita dall'applicazione.
In seguito alla selezione del template desiderato quest'ultimo deve essere visualizzato in un box dedicato e viene creato, in base ai dati editabili del template, un editor che ne permette la modifica per vedere come si comporta il template in base alla modifica di vari parametri.	\\
Per determinati template, come quelli considerati composti, l'editor deve dare la possibilità di visualizzare direttamente il JSON contenente i dati e permetterne la modifica.\\
Non essendo presente una struttura di beck-end proposta dall'applicazione Portal Studio, perché ancora in fase di valutazione, l'editor è stato sviluppato separatamente e il suo model consiste in un insieme di directory, contenenti i vari elementi che compongono i template, suddivise per categoria.\\
Per il caricamento dei template e dei loro dati vengono utilizzate chiamate \textbf{http} alle varie risorse, non essendo presenti delle API fornite dall'azienda.

