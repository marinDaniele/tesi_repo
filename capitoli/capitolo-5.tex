% !TEX encoding = UTF-8
% !TEX TS-program = pdflatex
% !TEX root = ../tesi.tex
% !TEX spellcheck = it-IT

%**************************************************************
\chapter{Analisi dei Requisiti}
\label{cap:analisi-dei-requisiti}
In questo capitolo sono contenuti i requisiti dell'applicazione che sono stati individuati durante il progetto.\\
Le linee guida per la creazione dell'applicazione sono state fornite dal tutor e sulla base di esse sono stati individuati i requisiti che in seguito sono stati dicussi con il tutor per ottenerne l'approvazione.
%**************************************************************

\section{Applicazione per la modifica dei template}
L'applicazione richiesta per il progetto deve permettere all'utente di visualizzare un insieme di template predefinito, da cui sia possibile selezionare quello desiderato.\\
In seguito alla selezione del template, l'utilizzatore deve poter visualizzare quest'ultimo all'interno di una \textit{view} apposita.\\
In questa fase deve essere fornito all'utente un editor specifico in relazione al template selezionato, che offra la possibilità di visualizzare e modificare i dati forniti di \textit{default} dal template e di vedere all'interno della \textit{view} dedicata il comportamento del template in seguito alla modifica dei dati.\\
L'applicazione deve eseguire all'interno di un browser e deve essere compatibile con i più importanti fra essi (Chrome, Firefox, Opera ed Edge).
\subsection{Visualizzazione lista dei template}
Questa sezione dell'applicazione è dedicata alla visualizzazione e selezione dei template disponibili.\\
La lista deve essere composta dalle miniature dei template in modo da offrire una prima visione di come viene rappresentato il template.\\
All'utente deve essere permesso di scorrere tutta la lista tramite uno scroll infinito.\\
La lista deve presentare i template per categorie, le seguenti sono quelle individuate durante l'analisi:
\begin{itemize}
	\item template singoli;
	\item template composti;
	\item template singoli contenenti plug-in JQuery;
	\item template composti contenenti plug-in JQuery;
\end{itemize}
I template \textit{singoli} sono dei template la cui costruzione è frutto dell'interpolazione dei dati con un singolo template, mentre quelli \textit{composti} sono template frutto dell'unione di più template.\\

\subsection{Visualizzazione template selezionato}

\subsection{Editor per la modifica del template}

\section{Requisiti individuati}

\section{Riepilogo requisiti}
