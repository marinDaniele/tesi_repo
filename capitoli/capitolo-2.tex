% !TEX encoding = UTF-8
% !TEX TS-program = pdflatex
% !TEX root = ../tesi.tex
% !TEX spellcheck = it-IT

%**************************************************************
\chapter{Librerie analizzate}
\label{cap:librerie-analizzate}
%**************************************************************
In questo capitolo vengono messe a confronto varie librerie \textbf{JavaScript} che permettono la realizzazione di template HTML, ne vengono analizzati i pregi e i difetti per arrivare a descrivere i motivi che hanno portato all scelta della libreria utilizzata nel progetto.

\section{Considerazioni generali}
Negli ultimi anni sono nate molte librerie che permettono la creazione di template HTML che hanno portato notevoli vantaggi agli sviluppatori, offrendo loro un nuovo strumento che permette di creare modelli HTML per la rappresentazione dei dati e riutilizzarli all'interno di pagine differenti con una considerevole diminuzione del codice JavaScript e HTML che normalmente viene utilizzato per la modifica de DOM.
Queste librerie si sono evolute velocemente fino ad arrivare a permettere agli sviluppatori di creare intere User interface per applicazioni web, creare componenti riutilizzabili ed in qualche caso offrire funzionalità avanzate come il two-way binding.

\subsection{I template con sintassi mustache}
Le librerie studiate durante lo stage utilizzano tutte questa particolare sintassi, che permette di rappresentare variabili, sezioni, parziali ed altri elementi utili alla creazione in maniera dinamica del template tramite l'inserimento di \textbf{tag}.\\
Questi particolari \textbf{tag} sono caratterizzati dall'utilizzo delle parentesi graffe come delimitatori e questo è il motivo per cui vengono definiti mustaches (baffi in inglese).\\
I tag si presentano nella forma "\{\{I P\}\}" dove I è un simbolo o una stringa ed identifica il tipo di tag, mentre P è un parametro o una chiave appartenente all'oggetto JSON correlato al template.\\
Per l'inserimento di variabili o parziali il \textbf{tag} è singolo, mentre per l'inserimento di sezioni, controlli del tipo not-exist ed altri sono presenti un \textbf{tag} di apertura ed uno di chiusura.\\
Per capire meglio il funzionamento che sta alla base di questi template engine è utile fare degli esempi.
\newpage
Questo esempio mostra il rendering di due variabili.
\begin{lstlisting}[language=JavaScript, caption=Esempio di template rappresentante una variabile.]
// oggetto JSON contenente i dati
var dati = { "name": "Jon", "age": 35};
// template HTML con l'aggiunta del tag mustache
var template = "<h1>Il mio nome è {{name}} e ho {{age}} anni.</h1>";

// il risultato del rendering sarà:

Il mio nome è Jon e ho 35 anni.
\end{lstlisting}
In questo esempio viene definito il template per rappresentare una lista di prodotti.
\begin{lstlisting}[language=JavaScript, caption=Esempio di template rappresentante una sezione.]
// oggetto JSON contenente i dati
var dati = prodotti: { 
						"prodotti": [
    							{ "name": "pane" },
    							{ "name": "pasta" },
    							{ "name": "biscotti" }
  						]
					};
// template HTML con l'aggiunta del tag mustache
var template = "<p>Lista della spesa:
					<ul>
						{{#prodotti}}
							<li>{{name}}</li>
						{{/prodotti}}
					</ul>
				</p>";

// il risultato del rendering sarà:

Lista della spesa:
- pane
- pasta
- biscotti
\end{lstlisting}

\section{Mustache.js}
Mustache può essere considerato come il papà dei template system, è open-souce e logic-less e presenta implementazioni per i più famosi linguaggi di programmaione, come Java, Phyton, Ruby, PHP, JavaScript e molti altri.\\
Mustache.js è un implementazione per JavaScript del template system Mustache.\\
La libreria è molto leggera e versatile visto che permette il rendering sia lato server che lato client.\\
Le funzioni offerte sono \textit{render} e \textit{parse}, la prima si occupano di creare la stringa HTML contenente il template renderizzato partendo dai dati JSON e dal template HTML e la seconda è opzionale e permette di preparare il template in modo da velocizzare l'operazione di render.\\
Mustache.js viene definita logic-less perché non presenta nessun tipo di costrutto \textit{if-then-else} e loop come \textit{for} o \textit{do-while}
\subsection{Come funziona}
Il suo funzionamento è molto semplice.
Per prima cosa bisogna includere la libreria all'interno della pagina HTML in cui si vuole inserire il template.
\begin{lstlisting}[language=HTML, caption=Inclusione libreria Mustache.js nel file HTML.]
<!doctype html>
<html>
  <head>
    ...
    <script type="text/javascript" src="mustache.js" ></script>
  </head>
  <body>
    ...
  </body>
</html>
\end{lstlisting}
In seguito all'inclusione basa richiamare la funzione \textit{render} passandogli i dati in formato JSON e il template HTML che dovrà visualizzarli.\\
La funzione restituisce una stringa HTML rappresentante il template renderizzato e per inserirla nella pagina è neccessario utilizzare funzioni JavaScript come \textit{innerHTML} o funzioni offerte dalla libreria JQuery perché Mustache.js non offre nessuno strumento per modificare il DOM.
\subsection{Pregi e difetti}

\subsection{Prototipo}


\FloatBarrier
\section{HandlebarsJS}

\subsection{Come funziona}

\subsection{Pregi e difetti}

\subsection{Prototipo}


\FloatBarrier
\section{Ractive.js}

\subsection{Come funziona}

\subsection{Pregi e difetti}

\subsection{Prototipo}


\clearpage
\section{Confronto finale}
%\todo[inline]{Trovare un nome migliore}

\FloatBarrier
\section{Libreria scelta}

