% !TEX encoding = UTF-8
% !TEX TS-program = pdflatex
% !TEX root = ../tesi.tex
% !TEX spellcheck = it-IT

%**************************************************************
\chapter{Librerie analizzate}
\label{cap:librerie-analizzate}
%**************************************************************
In questo capitolo vengono messe a confronto varie librerie \textbf{JavaScript} che permettono la realizzazione di template HTML, ne vengono analizzati i pregi e i difetti per arrivare a descrivere i motivi che hanno portato all scelta della libreria utilizzata nel progetto.

\section{Considerazioni generali}


\subsection{I template con sintassi mustache}

\section{Mustache.js}

\subsection{Come funziona}

\subsection{Pregi e difetti}

\subsection{Prototipo}


\FloatBarrier
\section{HandlebarsJS}

\subsection{Come funziona}

\subsection{Pregi e difetti}

\subsection{Prototipo}


\FloatBarrier
\section{Ractive.js}

\subsection{Come funziona}

\subsection{Pregi e difetti}

\subsection{Prototipo}


\clearpage
\section{Confronto finale}
%\todo[inline]{Trovare un nome migliore}

\FloatBarrier
\section{Libreria scelta}

