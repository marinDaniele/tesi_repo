% !TEX encoding = UTF-8
% !TEX TS-program = pdflatex
% !TEX root = ../tesi.tex
% !TEX spellcheck = it-IT

%**************************************************************
\chapter{Librerie analizzate}
\label{cap:librerie-analizzate}
%**************************************************************
In questo capitolo vengono messe a confronto varie librerie \textbf{JavaScript} che permettono la realizzazione di template HTML, ne vengono analizzati i pregi e i difetti per arrivare a descrivere i motivi che hanno portato all scelta della libreria utilizzata nel progetto.

\section{Considerazioni generali}
Negli ultimi anni sono nate molte librerie che permettono la creazione di template HTML che hanno portato notevoli vantaggi agli sviluppatori, offrendo loro un nuovo strumento che permette di creare modelli HTML per la rappresentazione dei dati e riutilizzarli all'interno di pagine differenti con una considerevole diminuzione del codice JavaScript e HTML che normalmente viene utilizzato per la modifica de DOM.
Queste librerie si sono evolute velocemente fino ad arrivare a permettere agli sviluppatori di creare intere User interface per applicazioni web, creare componenti riutilizzabili ed in qualche caso offrire funzionalità avanzate come il two-way binding.

\subsection{I template con sintassi mustache}
Le librerie studiate durante lo stage utilizzano tutte questa particolare sintassi formata da tag, che permettono di rappresentano variabili, sezioni, parziali ed altri elementi utili alla creazione in maniera dinamica del template.
Questi particolari tag sono caratterizzati dall'utilizzo delle parentesi graffe come delimitatori e questo è il motivo per cui vengono definiti mustaches (baffi in inglese).
Si presentano nella forma "{{...}}" e possono contenere al loro interno un simbolo, che identifica il tipo di tag e un parametro o una chiave appartenente all'oggetto JSON correlato al template.
Questi tag possono essere singoli, come nel caso delle variabili o doppi, cioè nella forma tag di apertura e tag di chiusura, come nel caso delle sezioni.
Per capire bene il funzionamento che sta alla base di questi template engine è utile fare degli esempi.
\begin{lstlisting}[language=JavaScript, caption=Esempio di template rappresentante una variabile.]
var view = { name: "Jon", age: 35};

var template = "<h1>My name is {{name}} and i am {{age}} years old.</h1>";

// il risultato del rendering sarà:

My name is Jon and i am 35 years old.
\end{lstlisting}
In questo esempio viene definito il template per rappresentare una lista di prodotti.
\begin{lstlisting}[language=JavaScript, caption=Esempio di template rappresentante una sezione.]
var view = prodotti: { 
						"prodotti": [
    							{ "name": "pane" },
    							{ "name": "pasta" },
    							{ "name": "biscotti" }
  						]
					};

var template = "Lista della spesa:
					<ul>
						{{#prodotti}}
						<li>{{name}}</li>
						{{/prodotti}}
					</ul>";

// il risultato del rendering sarà:

Lista della spesa:
- pane
- pasta
- biscotti
\end{lstlisting}

\section{Mustache.js}

\subsection{Come funziona}

\subsection{Pregi e difetti}

\subsection{Prototipo}


\FloatBarrier
\section{HandlebarsJS}

\subsection{Come funziona}

\subsection{Pregi e difetti}

\subsection{Prototipo}


\FloatBarrier
\section{Ractive.js}

\subsection{Come funziona}

\subsection{Pregi e difetti}

\subsection{Prototipo}


\clearpage
\section{Confronto finale}
%\todo[inline]{Trovare un nome migliore}

\FloatBarrier
\section{Libreria scelta}

