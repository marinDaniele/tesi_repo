% !TEX encoding = UTF-8
% !TEX TS-program = pdflatex
% !TEX root = ../tesi.tex
% !TEX spellcheck = it-IT

%**************************************************************
\chapter{I template}\label{cap:template}
In questo capitolo viene descritto il lavoro svolto nella prima parte del progetto, che consiste nella realizzazione dei template, nel spiegare come sono stati resi responsive ed in fine come sia stato possibile inserire \textit{plug-in} JQuery al loro interno.\\
Viene anche trattato l'argomento relativo al loro caricamento all'interno della pagina HTML e la gestione delle librerie.
%**************************************************************
\section{Utilizzo di Ractive.js}
In questa sezione viene descritta più in dettaglio la libreria scelta per svolgere il progetto.
\subsection{L'oggetto Ractive}

\subsection{Le opzioni principali}

\subsection{La sintassi mustaches}

\subsection{Il two-way data biding}

\subsection{Gli eventi}

\subsection{Il virtual DOM}

\subsection{Creazione di un template}

\subsection{Plug-in di terze parti}\label{sec:packager}

\section{Struttura dei template}

\section{Rendere il template responsive}

\section{Inserimento plug-in jQuery nei template}

\section{Caricamento dei template nelle pagine HTML}

