% !TEX encoding = UTF-8
% !TEX TS-program = pdflatex
% !TEX root = ../tesi.tex
% !TEX spellcheck = it-IT

%**************************************************************
\chapter{Progettazione}
\label{cap:progettazione}
In questo capitolo viene spiegato il metodo di suddivisione delle risorse, come viene effettuato il caricamento dei vari componenti dei template e vengono descritte le funzioni che compongono l'applicazione.
%**************************************************************
\section{Suddivisione template}
Come descritto nel capitolo \textit{Analisi dei requisiti} i template sono stati divisi in quattro tipologie.\\
Dato che non sono presenti delle API fornite dall'azienda, per ottenere informazioni sui template disponibili, come il loro numero, il loro tipo e le risorse che li compongono, è stato deciso di utilizzare un metodo semplice per gestire i template.\\
Questo metodo consiste nel distinguere i vari template tramite il loro nome.\\
Il nome di ogni template è composto da due parti, la prima definisce il tipo e la seconda è un numero che identifica ogni singolo template.\\
Le quattro categorie vengono nominate come segue:
\begin{itemize}
	\item \textbf{tml} identifica i template semplici;
	\item \textbf{jtml} identifica i template semplici contenenti plug-in JQuery;
	\item \textbf{ctml} identifica i template composti;
	\item \textbf{jctml} identifica i template composti contenenti plug-in JQuery.
\end{itemize}
Quindi il nome \textbf{tml1} identificherà il primo dei template semplici all'interno della directory \textit{templates}.\\
I file che definiscono il template utilizzano lo stesso nome del template a cui appartengono.

\section{Caricamento template}
\section{Creazione lista template}
\section{Visualizzazione template selezionato}
\section{Editor per la modifica del template}