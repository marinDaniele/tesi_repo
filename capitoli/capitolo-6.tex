% !TEX encoding = UTF-8
% !TEX TS-program = pdflatex
% !TEX root = ../tesi.tex
% !TEX spellcheck = it-IT

%**************************************************************
\chapter{Progettazione}
\label{cap:progettazione}
In questo capitolo viene spiegato il metodo di suddivisione delle risorse, come viene effettuato il caricamento dei vari componenti dei template e vengono descritte le funzioni che compongono l'applicazione.
%**************************************************************
\section{Suddivisione template}
Come descritto nel capitolo \textit{Analisi dei requisiti} \ref{subsec:visualizzazione_template} i template sono stati divisi in quattro tipologie.\\
Dato che non sono presenti delle API fornite dall'azienda, per ottenere informazioni sui template disponibili, come il loro numero, il loro tipo e le risorse che li compongono, è stato deciso di utilizzare un metodo semplice per gestire i template all'interno del file-system.\\
Le risorse sono state suddivise in directory, che rappresentano i template.\\
Ad ogni directory e alle risorse principali in esse contenute è stato assegnato un nome che permette di identificare il singolo template e il suo tipo.\\
Le quattro categorie vengono nominate come segue:
\begin{itemize}
	\item \textbf{tml} identifica i template di tipo semplici;
	\item \textbf{jtml} identifica i template di tipo semplici contenenti plug-in jQuery;
	\item \textbf{ctml} identifica i template di tipo composti;
	\item \textbf{jctml} identifica i template di tipo composti contenenti plug-in jQuery.
\end{itemize}
\subsection{Accesso alle risorse}
Le risorse necessarie sono reperibili tramite chiamata \textit{http-get} ad un URL nella forma \texttt{/templates/\{tipo+indice\}/\{tipo+indice\}.estensione}, dove \texttt{tipo} è uno dei quattro tipi descritti in precedenza, \texttt{indice} è il numero che identifica il template.

\newpage
\subsubsection{GET /templates/\{tipo+indice\}/\{tipo+indice\}.html}
Restituisce il file HTML che descrive il template, nel caso in cui l'operazione non vada a buon fine viene lanciato un errore di tipo \texttt{404} e un messaggio \textit{file non trovato}.
\begin{lstlisting}[language=HTML, caption=Esempio di template mustache restituito.]
<div id="tml1" style="background-color: {{bgColor}}; color: {{textColor}};">

	<img class="image" src= {{ img ? img : 'templates/tml1/img/no-image.png'}}>
	<div class="prod-data">
		<h2 class="center">{{nome}}</h2>
		prezzo: <strong>{{prezzoCent}} Euro</strong>
		<p class={{ quantita ? 'disponibile' : 'non-disponibile'}} >Disponibilità: {{#if quantita}}
			<strong>disponibili {{quantita}}</strong>
			{{else}}
			<strong>Non disponibile</strong>
			{{/if}}
		</p>
		<h3>Descrizione:</h3>
		<p class="desc">{{descrizione}}</p>
	</div>
</div>

<!-- stili -->
<style>
#tml1{
	position: relative;
	font-family: Arial, Verdana, Helvetica, sans-serif;
	border: 1px solid black;
	overflow: auto;
}

...

</style>

<!-- comportamento -->
<script>
	component.exports = {
		computed: {
			prezzoCent: function(){
				return this.get('prezzo').toFixed(2);
			}
		}
		
		...
	}
</script>
\end{lstlisting}

\newpage
\subsubsection{GET /templates/\{tipo+indice\}/\{tipo+indice\}.json}
Restituisce l'oggetto JSON contenente i dati del template, nel caso in cui l'operazione non vada a buon fine viene lanciato un errore di tipo \texttt{404} e un messaggio \textit{file non trovato}.

\begin{lstlisting}[language=JSON, caption=Esempio di oggetto JSON restituito.]
{
	"bgColor":string,
	"textColor":string,
	"nome": string,
	"img": string,
	"prezzo": number,
	"quantita": number
}
\end{lstlisting}
\subsubsection{GET /templates/\{tipo+indice\}/\{tipo+indice\}\_libs.json}
Restituisce, se il template contiene plug-in, l'oggetto JSON contenente l'URL delle librerie utilizzate, nel caso in cui l'operazione non vada a buon fine viene lanciato un errore di tipo \texttt{404} e un messaggio \textit{file non trovato}.
\begin{lstlisting}[language=JSON, caption=Esempio di oggetto JSON restituito.]
{
	"libs": [
		string,
		string,
		...
	]
}
\end{lstlisting}

\section{Caricamento template}
Il caricamento di un template consiste nel reperire le risorse principali ad esso relative e con queste istanziare un oggetto di tipo ractive, che una volta creato verrà renderizzato in un apposito elemento HTML.\\
Per effettuare questa operazione sono state definite le seguenti funzioni:
\begin{itemize}
	\item \texttt{loadTemplateWithoutJQuery(tmlUrl, dataUrl, tmlAnchor)}\\
	Questa funzione si occupa del caricamento e del rendering dei template che \textit{non contengono} plug-in jQuery.\\
	La funzione recupera le risorse da \texttt{tmlUrl} e \texttt{dataUrl}, crea un'istanza \textit{Ractive} che rappresenta il template e lo renderizza all'interno dell'elemento HTML con id \texttt{tmlAnchor}.
	
	\item \texttt{loadTemplateWithJQueryPlugins(tmlUrl, dataUrl, libsUrl, tmlAnchor)}\\
	Questa funzione si occupa del caricamento e del rendering dei template che \textit{contengono} plug-in jQuery.\\
	La funzione recupera la lista delle librerie jQuery del template da \texttt{libsUrl} e usa la funzione \texttt{loadLibs} per gestire il loro caricamento.\\
	In seguito recupera le risorse da \texttt{tmlUrl} e \texttt{dataUrl}, crea un'istanza \textit{Ractive} che rappresenta il template e lo renderizza all'interno dell'elemento HTML con id \texttt{tmlAnchor}.
	
	\newpage
	\item \texttt{loadLibs(libs)}\\
	Questa funzione riceve un \textit{array} di URL relative alle librerie jQuery da caricare e utilizzando la funzione \texttt{addLibraryFromUrl(libUrl)} ne gestisce il caricamento.
	\begin{lstlisting}[language=JSON, caption=Esempio array \texttt{libs}.]
	{
		"libs": [
			"templates/template_name/lib/library1_name.js",
			"templates/template_name/lib/library2_name.js",
			"templates/template_name/lib/library3_name.js"
		]
	}
	\end{lstlisting}
	
	\item \texttt{addLibraryFromUrl(libUrl)}\\
	Questa funzione provvede al caricamento della libreria presente all'URL \texttt{libUrl} nel caso questa non sia stata caricata.
	
\end{itemize}
\section{Creazione lista template}
In questa sezione vengono descritte le funzioni che si occupano della creazione della lista di selezione dei template.\\
Questa lista deve essere composta dalle miniature dei template suddivise fra le quattro categorie descritte in precedenza.\\
Per poter effettuare il caricamento è necessario definire un oggetto JavaScript che contenga il numero di template disponibili suddivisi per categoria.\\
La scelta di definire questo oggetto è stata dettata dalla necessità di conoscere l'insieme dei template disponibili.\\
Data la struttura del file-system e il fatto che i template sono indicizzati in maniera ordinata è possibile descrivere l'insieme dei template da caricare tramite un oggetto JavaScript.\\

\begin{itemize}
	\item \texttt{templatesToLoad} \label{ttlObject} \\
	Questo oggetto descrive l'insieme dei template offerti dall'applicazione.\\
	La sua struttura è formata da coppie del tipo \textit{chiave-valore} che rappresentano rispettivamente:
	\begin{itemize}
		\item \textbf{chiave} definisce il tipo dei template;
		\item \textbf{valore} contiene il numero dei template di quel tipo.
	\end{itemize}
	L'oggetto è rappresentato come segue:
	\begin{lstlisting}[language=JSON, caption=Struttura oggetto \texttt{templatesToLoad}.]
	{
		"tml": number,
		"jtml": number,
		"ctml": number,
		"jctml": number
	}
	\end{lstlisting}
	
	\item \texttt{createTemplateList()}\\
	Questa funzione si occupa di inserire le varie categorie all'interno della lista e tramite l'ausilio della funzione \texttt{loadTemplateList(type, num)} di popolare la lista.\\
	La funzione utilizza l'oggetto \texttt{templateToLoad} per conoscere l'insieme dei template.
	
	\item \texttt{loadTemplateList(type, num)}\\
	Questa funzione si occupa di creare la miniatura e inserirla nella lista in base ai parametri \texttt{type} e \texttt{num} che compongono il nome del template.
\end{itemize}

\section{Visualizzazione template selezionato}
In questa sezione viene descritto il comportamento previsto e le funzioni individuate per la gestione del \textit{view-box} che andrà a visualizzare il template selezionato.\\
Alla base del \textit{view-box} c'è un elemento HTML di tipo \texttt{div} con id \texttt{"tml-anchor"} che farà da contenitore per i template.\\
Ogni miniatura presente nella lista in seguito all'evento \texttt{click()} invoca una funzione che si occupa di visualizzare il template nel \textit{view-box}.
\begin{itemize}
	\item \texttt{selectTml(el)}\\
	Questa funzione riceve il nome del template da caricare dal parametro \texttt{el} ed in seguito si occupa di effettuare il caricamento delle risorse e il rendering del template nel \texttt{div} prestabilito.
\end{itemize}
\section{Editor per la modifica del template}
In seguito alla selezione di un template questo deve essere visualizzato e sulla base dei suoi dati modificabili deve essere costruito un editor che ne permetta la modifica.\\
In questa sezione vengono descritte le funzioni che si occupano di creare l'editor per la modifica dei dati del template.\\
La creazione dell'editor nelle sue diverse versioni (per template semplici e composti) viene gestita dalle seguenti funzioni:
\begin{itemize}
	\item \texttt{creaTml(type,dati)}\\
	Questa funzione riceve il tipo del template selezionato tramite il parametro \texttt{type} e l'oggetto JSON contenente i dati del tramite il parametro \texttt{dati}.\\
	La funzione costruisce l'editor in base al tipo del template, utilizzando le funzioni \texttt{creaEditorFullJson( dati, editorAnchor )} e \\ \texttt{creaEditor( dati, editorAnchor )} che verranno descritte in seguito.
	
	\item \texttt{creaEditorFullJson(obj, el)}\\
	Questa funzione è rivolta ai template di tipo \textit{composto} e si occupa di creare un area editabile contenete il codice che descrive l'oggetto JSON fornito con il parametro \texttt{obj} e ne permette la modifica.\\
	Gli elementi creati dalla funzione vengono inseriti nell'elemento HTML con id \texttt{el}.
	
	\newpage
	\item \texttt{creaEditor(obj, el)}\\
	Questa funzione è rivolta ai template di tipo \textit{semplice} e si occupa di creare l'editor contenente tutti gli strumenti per la modifica di ogni dato all'interno del parametro \texttt{obj}.\\
	L'editor viene inserito all'interno dell'elemento HTML con id \texttt{el} indicando il nome del dato seguito dallo strumento per la sua modifica.\\
	Durante la creazione dell'editor viene utilizza la funzione \texttt{parse} per la costruzione dei vari strumenti che andranno a comporre l'editor.
	
	\item \texttt{parse( obj, el, index, regExpArray, path )}\\
	Questa funzione riceve l'oggetto JSON tramite il parametro \texttt{obj} e utilizzando i parametri \texttt{index}, \texttt{regExpArray} e \texttt{path} come controllo, crea lo strumento più adeguato per ogni tipo di dato individuato aggiungendo ognuno di essi all'elemento HTML con id \texttt{el}.\\
	Gli strumenti creati dalla funzione offrono la possibilità di modifica per i dati di tipo:
	\begin{itemize}
		\item \textbf{colore} tramite color picker;
		\item \textbf{numero} tramite input number;
		\item \textbf{booleano} tramite combobox (true,false);
		\item \textbf{immagine} tramite selettore di immagini da file-system;
		\item \textbf{url} tramite line-edit;
		\item \textbf{mail} tramite line-edit;
		\item \textbf{sringa breve} tramite line-edit;
		\item \textbf{testo} tramite text-area;
	\end{itemize}
\end{itemize}