% !TEX encoding = UTF-8
% !TEX TS-program = pdflatex
% !TEX root = ../tesi.tex
% !TEX spellcheck = it-IT

%**************************************************************
\chapter{Realizzazione}
\label{cap:realizzazione}
In questo capitolo vengono descritte le attività svolte durante lo sviluppo dell'applicazione e le principali difficoltà riscontrate.\\
Per lo sviluppo dell'applicazione sono state utilizzate, riadattandole, anche soluzioni sviluppate durante la fase di studio sui template, in particolare quelle relative al caricamento dei template e alla gestione delle librerie JQuery.\\
Durante questa fase il lavoro svolto è stato proposto al tutor aziendale in più riprese perché ne verificasse il comportamento e proponesse eventuali modifiche.
%**************************************************************

\section{Il caricamento dei template}
Il caricamento dei template consiste in tre fasi, che sono:
\begin{itemize}
	\item caricamento delle risorse;
	\item creazione istanza \textit{Ractive};
	\item rendering del template all'interno di un elemento HTML.
\end{itemize}
Per effettuare il caricamento delle risorse sono stati utilizzati due metodi differenti in base al tipo di file da caricare.\\
Il caricamento degli oggetti di tipo JSON, come i dati e l'elenco delle librerie JQuery del template, è stato effettuato tramite un metodo offerto dalla libreria JQuery, che permette di effettuare una \textit{GET HTTP request} per il caricamento specifico di oggetti JSON.\\
Il metodo in questione è \href{http://api.jquery.com/jquery.getjson/}{\texttt{jQuery.getJSON()}} che effettua una \textit{callback} ad un URL e ritorna l'oggetto desiderato.\\
Per effettuare il caricamento del template mustache invece, la comunità di Ractive offre un plug-in chiamato \textit{ractive-load} che aggiunge un metodo statico alla libreria e permette, tramite la \href{http://www.ecma-international.org/ecma-262/6.0/#sec-promise-objects}{\textit{promise}} \href{https://github.com/ractivejs/ractive-load}{\texttt{Ractive.load()}} di effettuare il caricamento del file contenente la definizione del template utilizzando \textit{GET HTTP request}.
\newpage
\begin{lstlisting}[language=JavaScript, caption=Chiamate \textit{GET HTTP} per il caricamento delle risorse.]
// carico i dati del template
$.getJSON(dataUrl, function(dati) { // se il caricamento ha successo
	// carico il template tramite Ractive.load
	Ractive.load(tmlUrl).then( function(Template) {
		// creo l'oggetto ractive
		var ractive = new Template({
			el: tmlAnchor,
			data: dati
		});
	...

	});
})
.fail( function() { // errore caricamento, file non valido
	console.log('file non trovato o errore di caricamento!');
});
\end{lstlisting}
Le richieste vengono eseguite in modo asincrono, quindi solamente l'esito positivo della prima \textit{callback} permette l'esecuzione della chiamata a \texttt{Ractive.load()} e l'eventuale istanziazione dell'oggetto \textit{Ractive}.\\
Per quanto riguarda il caricamento di template contenenti plug-in JQuery, il metodo è identico, ma prima di caricare i dati ed il template devono essere caricate le librerie.\\
Il caricamento delle librerie viene effettuato tramite \texttt{JQuery.getJSON()} del file contenente la lista delle librerie e aggiungendo le URL di quest'ultime all'\texttt{header} della pagina dell'applicazione tramite la creazione di un tag \texttt{script} per ogni libreria individuata.\\
\begin{lstlisting}[language=JavaScript, caption=Esempio aggiunta librerie all'HTML dell'applicazione.]
// carico le librerie del template
$.getJSON(libsUrl, function(libs) { // se il caricamento ha successo
	// aggiungo le librerie alla pagina HTML
	scriptControll.loadLibs(libs);

	// carico i dati del template
	$.getJSON(dataUrl, function(dati) { // se il caricamento ha successo
		...
		// istanziazione oggetto ractive
})
.fail( function() { // librerie non trovate o errore di caricamento
	console.log('file non trovato o errore di caricamento!');

// esempio oggetto JSON delle librerie
{
	"libs": [
		"templates/jtml1/lib/actuate-animate.min.js",
		"templates/jtml1/lib/jquery.drawsvg.min.js"
	]
}
// esempio aggiunta script all'HTML
<head>
		...
	<!-- script aggiunti -->
	<script src="templates/jtml1/lib/actuate-animate.min.js"></script>
	<script src="templates/jtml1/lib/jquery.drawsvg.min.js"></script>

</head>

\end{lstlisting}
Per i template con plug-in JQuery il problema principale è quello che il template venga renderizzato prima del caricamento delle librerie, questo comporta il non riconoscimento delle funzioni che si riferiscono al plug-in rendendo il template incompleto.\\
Quindi il caricamento delle librerie viene eseguito sempre prima di caricare gli altri elementi del template.\\
Nonostante questa accortezza risulta impossibile verificare l'effettivo caricamento da parte del \textit{browser} delle librerie perché esso viene effettuato in modo asincrono.\\
Questo problema può essere risolto in maniera semplice con un \textit{reload} della pagina o come è stato fatto in un fork dell'applicazione tramite un \textit{preload} di tutte le librerie, che però risulta una soluzione molto onerosa e in certi casi non risolve il problema.

\section{Controllo delle librerie caricate}


\section{Visualizzatore lista template}

\section{Visualizzatore template selezionato}

\section{Editor per la manipolazione del template}
